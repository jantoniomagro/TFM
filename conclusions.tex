\chapter{Conclusions and future work} % (fold)
\label{cha:conclusions_and_future_work}

\section{Conclusions} % (fold)
\label{sec:conclusions}

\begin{itemize}

\item Relational approach are not always suitable for any problem. Non-relational systems are not cure-all, but it has been shown they can face some problems in a more efficient way (\textit{e.g.} MapReduce) and, in some situations, NoSQL can be a complement for existing RDBMS.

\item NoSQL database systems, specially -not exclusively- those document-oriented can reduce system analysis and design and can also succeed in boosting the performance not just of data querying ---their main goal---, but also simplify data management and data versioning, by simplifying ingestion and data updates.

\item It is possible to convert an existing VO framework, such as the OpenCADC (in the configuration used for the ALMA Science Archive), without touching but a few classes. This gives VO tools access to NoSQL capabilities, without having to touch the user-exposed VO interfaces.

\end{itemize}

% section conclusions (end)

\section{Future work} % (fold)
\label{sec:future_work}

\begin{itemize}

% NOTA: elimino este punto porque no lo tengo claro
% \item Focusing in a workgroup inside Virtual Observatory instead of treating several aspects.

\item Finish a reference NoSQL implementation of the OpenCADC.

\item % The
Use % of
formal
software engineering
metrics
(CoCoMo, Function Point Analysis, etc.) to
evaluate % plan
the
OpenCADC
redesign
and current and future costs involved.

\item Benchmark the OpenCADC NoSQL system
in order to obtain accurate data in performance improvements.

\end{itemize}

% section future_work (end)

% chapter conclusions_and_future_work (end)

