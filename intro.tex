\chapter{Introduction}

% NOTA: No es buena idea empezar tan pronto con un ejemplo.

The volume of data produced at
% some
science centers presents a
%considerable
processing challenge
that is getting more and more difficult to deal with.
At the European Organization for Nuclear Research\urlnote{http://home.web.cern.ch}
(CERN), for instance,
the Large Hadron Collider\urlnote{http://home.web.cern.ch/about/accelerators/large-hadron-collider} (LHC) will record
several hundred million collision events,
which are recorded so that physicists can match them against simulated events (again, millions of them), and then determine if the collisions have thrown up any interesting information.

CERN, like many other science centers, does not have the computing or financial resources to manage all of the data
on site,
so it moved into the grid to share the load with computing centers around the world. The Worldwide LHC Computing Grid is a distributed system which gives over 8000 physicists near real-time access to LHC data.

Other current or future astronomical  facilities such as the Low-Frequency Array\urlnote{http://www.lofar.org} (LOFAR), the Australian Square Kilometre Array Pathfinder\urlnote{http://www.atnf.csiro.au/projects/mira/} (ASKAP), the Large Synoptic Survey Telescope\urlnote{http://www.lsst.org/lsst/} (LSST), or the Square Kilometre Array\urlnote{http://www.skatelescope.org/} (SKA), will also be delivering much more data that it is feasible for the community to directly download or process. Tools for making
it
easier for astronomers to operate on those larger datasets are needed, and indeed are priorities of those
%facilities.
facilities, as we are moving to a data-intensive paradigm~\cite{Micro_01}, in which large datasets are driving the way we derive meaning in all scientific fields.

% NOTA: Hablar de obsolescencia en este párrafo es demasiado pronto; mejor ir justificando los problemas de los RDBMS, y por qué puede convenir algo distinto.
% So we could think that current deployed technologies are maybe obsolete and a new rethink should be made. If CERN (but not just CERN as we will discuss later) has changed its storage policy (leaving behind the classic client-server model), why not change the way the data are accessed? 

However, current facilities are also suffering from the lack of dedicated tools for astronomers, as their data and metadata sizes are also increasing beyond what can be comfortably manipulated and downloaded from the personal workstations of scientists. For instance, the datasets from the Atacama Large Millimetre and Submillimetre Array\urlnote{http://www.almaobservatory.org/} (ALMA), the largest radio interferometer currently in operations, typically occupy several gigabytes in their current configuration, and will become larger as more and more antennas and receivers are % added.
added, increasing the telescope sensibility and the number of available baselines.

In order for observatories like ALMA to store and publish their observations, they have typically relied on relational database management systems (RDBMS; see~\cite{Silberschatz_01}). However, RDBMS have their weaknesses, specially for data publishing applications, as they impose the same schema for datasets which might not have all of their metadata in common, and mandate an ingestion phase that converts the metadata in the original format into a format suitable for RDBMS systems.

Today, non-relational, cloud, or the so-called NoSQL (\emph{Not-only-SQL}) database systems
are growing rapidly as an alternative model for database management. These systems are tuned for the kind of big data applications that have made possible very large systems such as Google or Facebook, and can incorporate the documents themselves, and query on the existing metadata, without the need for a dedicated, complicated ingestion phase.

% NOTA: Esto no es necesario, porque además el tribunal tampoco lo va a entender.
% Often more features apply such as schema-less, easy replication feature, simple API, eventually consistent / BASE (not ACID), a huge amount of data (big data) and more. 

In this work, we give a short overview of
the
big data 
challenges being faced by astronomy, % problem
and present an alternative, using one of the
freely available % cambio de orden
NoSQL databases, 
% NOTA: Estoy no es cierto del todo. Depende de qué funciones de un RDBMS utilicemos
% (today, the volumes of data that can be handled by NoSQL systems exceed what can be handled by the biggest RDBMS)
and how it can be integrated in the Virtual Observatory framework.