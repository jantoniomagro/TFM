\chapter{Overview}

\pagenumbering{arabic}

The volume of data produced at some science centers presents a considerable processing challenge.\newline

At CERN, for instance, almost 600 million times per second, particles collide within the Large Hadron Collider (LHC). Each collision generates particles that often decay in complex ways into even more particles. Electronic circuits record the pass of particles through a detector and send the data for digital reconstruction. The digitized summary is recorded as a "collision event". Physicists must sift through the 15 petabytes or so of data produced annually to determine if the collisions have thrown up any interesting physics.  \newline

CERN, like many other science centers, does not have the computing or financial resources to manage all of the data on its site, so it turned to grid computing to share the burden with computing centers around the world. The Worldwide LHC Computing Grid is a distributed system which gives over 8000 physicists near real-time access to LHC data.  \newline

So we could think that current deployed technologies are maybe obsolete and a new rethink should be made. If CERN (but not just CERN as we will discuss later) has changed its storage policy (leaving behind the classic client-server model), why not change the way the data are accessed?  \newline

Almost a decade after E. Codd published his famous relational model paper, the relational database management systems (RDBMS) have been the \textit{de facto} tools. Today, non-relational, cloud, or the so-called NoSQL databases are growing rapidly as an alternative model for database management. Often more features apply such as schema-free, easy replication support, simple API, eventually consistent / BASE (not ACID), a huge amount of data (big data) and more. \newline

In this work, we present an alternative, using one of the NoSQL databases (today, the volumes of data that can be handled by NoSQL systems exceed what can be handled by the biggest RDBMS) free available, to offer a different way of challenging these problems, always operating inside VO frame.