\documentclass[oneside,numbers,english]{ezthesis}

\usepackage{graphicx}
\usepackage{url}
\usepackage{float}
\usepackage{fancybox}

\usepackage{framed,color}

\usepackage{listings}
\usepackage{color}

\definecolor{dkgreen}{rgb}{0,0.6,0}
\definecolor{gray}{rgb}{0.5,0.5,0.5}
\definecolor{mauve}{rgb}{0.58,0,0.82}
\definecolor{shadecolor}{rgb}{1,0.8,0.3}

\lstset{frame=tb,
  language=Java,
  aboveskip=3mm,
  belowskip=3mm,
  showstringspaces=false,
  columns=flexible,
  basicstyle={\small\ttfamily},
  numbers=none,
  numberstyle=\tiny\color{gray},
  keywordstyle=\color{blue},
  commentstyle=\color{dkgreen},
  stringstyle=\color{mauve},
  breaklines=true,
  breakatwhitespace=true
  tabsize=2
}


%% # Opciones disponibles para el documento #
%%
%% Las opciones con un (*) son las opciones predeterminadas.
%%
%% Modo de compilar:
%%   draft            - borrador con marcas de fecha y sin im'agenes
%%   draftmarks       - borrador con marcas de fecha y con im'agenes
%%   final (*)        - version final de la tesis
%%
%% Tama'no de papel:
%%   letterpaper (*)  - tama'no carta (Am'erica)
%%   a4paper          - tama'no A4    (Europa)
%%
%% Formato de impresi'on:
%%   oneside          - hojas impresas por un solo lado
%%   twoside (*)      - hijas impresas por ambos lados
%%
%% Tama'no de letra:
%%   10pt, 11pt, o 12pt (*)
%%
%% Espaciado entre renglones:
%%   singlespace      - espacio sencillo
%%   onehalfspace (*) - espacio de 1.5
%%   doublespace      - a doble espacio
%%
%% Formato de las referencias bibliogr'aficas:
%%   numbers          - numeradas, p.e. [1]
%%   authoryear (*)   - por autor y a'no, p.e. (Newton, 1997)
%%
%% Opciones adicionales:
%%   spanish         - tesis escrita en espa'nol
%%
%% Desactivar opciones especiales:
%%   nobibtoc   - no incluir la bibiolgraf'ia en el 'Indice general
%%   nofancyhdr - no incluir "fancyhdr" para producir los encabezados
%%   nocolors   - no incluir "xcolor" para producir ligas con colores
%%   nographicx - no incluir "graphicx" para insertar gr'aficos
%%   nonatbib   - no incluir "natbib" para administrar la bibliograf'ia

%% Paquetes adicionales requeridos se pueden agregar tambi'en aqu'i.
%% Por ejemplo:
%\usepackage{subfig}
%\usepackage{multirow}

%% # Datos del documento #
%% Nota que los acentos se deben escribir: \'a, \'e, \'i, etc.
%% La letra n con tilde es: \~n.

\author{Jos\'e Antonio Magro Cort\'es}
\title{Integrating NoSQL Technologies into Virtual Observatory}
\degree{Master's Thesis}
\supervisor{Juan de Dios Santander Vela}
\institution{Universidad de Granada / Instituto de Astrof\'isica de Andaluc\'ia}
\faculty{}
\department{}

%% # M'argenes del documento #
%% 
%% Quitar el comentario en la siguiente linea para austar los m'argenes del
%% documento. Leer la documentaci'on de "geometry" para m'as informaci'on.

%\geometry{top=40mm,bottom=33mm,inner=40mm,outer=25mm}

%% El siguiente comando agrega ligas activas en el documento para las
%% referencias cruzadas y citas bibliogr'aficas. Tiene que ser *la 'ultima*
%% instrucci'on antes de \begin{document}.
\hyperlinking
\begin{document}

%% En esta secci'on se describe la estructura del documento de la tesis.
%% Consulta los reglamentos de tu universidad para determinar el orden
%% y la cantidad de secciones que debes de incluir.

%% # Portada de la tesis #
\include{titlepage}


\tableofcontents

%% # Prefacios #
%% Por cada prefacio (p.e. agradecimientos, resumen, etc.) crear
%% un nuevo archivo e incluirlo aqu'i.
%% Para m'as detalles y un ejemplo mirar el archivo "gracias.tex".
%\prefacesection{Acknowledgments}


\pagenumbering{Roman}
\begin{flushright}

\textit{En primer lugar debo reconocer la deuda que tengo con mi tutor, el Dr. Juande Santander, por aceptarme como alumno, por compartir sus conocimientos conmigo, por sus revisiones, anotaciones y correcciones. Por su disponibilidad a pesar de la distancia, de la diferencia horaria y del resto de circunstancias. Por la pasi\'on que le pone a su trabajo. !`Gracias maestro!}


\textit{\newline A la Dra. Maribel Tercedor, porque sin ella preverlo, hace 15 a\~{n}os una frase suya cambi\'o mi rumbo laboral. Este bandazo acad\'emico se lo debo a ella.}


\textit{\newline A Mamen, Antonio y Alejandro por permitir embarcarme en esta locura a pesar de las horas que les iba a robar y por aportar la cordura necesaria para compatibilizar trabajo, estudios y familia.}


\textit{\newline A mis padres, por contagiarme el virus de la curiosidad a base de comprarme libros y m\'as libros y darme toda clase de facilidades.}


\textit{\newline A mis abuelos: el \'exito de vuestra lucha se refleja en momentos como \'este, a pesar de que \'ultimamente algunos se empe\~{n}an en destruir vuestro legado. No lo conseguir\'an.}


\textit{\newline A Laura y Mar\'ia del Mar, compa\~{n}eras, por las risas entre clase y clase (y fuera de ellas) y por haber ayudado a este \emph{intruso}. Llegar\'eis lejos, os lo merec\'eis.} 

\end{flushright}



%\begin{abstract} Abstract. \end{abstract}


%% # 'Indices y listas de contenido #
%% Quitar los comentarios en las lineas siguientes para obtener listas de
%% figuras y cuadros/tablas.
%\tableofcontents
%\listoffigures
%\listoftables

%% # Cap'itulos #
\chapter{Overview}

\pagenumbering{arabic}

The volume of data produced at some science centers presents a considerable processing challenge.\newline

At CERN, for instance, almost 600 million times per second, particles collide within the Large Hadron Collider (LHC). Each collision generates particles that often decay in complex ways into even more particles. Electronic circuits record the pass of particles through a detector and send the data for digital reconstruction. The digitized summary is recorded as a "collision event". Physicists must sift through the 15 petabytes or so of data produced annually to determine if the collisions have thrown up any interesting physics.  \newline

CERN, like many other science centers, does not have the computing or financial resources to manage all of the data on its site, so it turned to grid computing to share the burden with computing centers around the world. The Worldwide LHC Computing Grid is a distributed system which gives over 8000 physicists near real-time access to LHC data.  \newline

So we could think that current deployed technologies are maybe obsolete and a new rethink should be made. If CERN (but not just CERN as we will discuss later) has changed its storage policy (leaving behind the classic client-server model), why not change the way the data are accessed?  \newline

Almost a decade after E. Codd published his famous relational model paper, the relational database management systems (RDBMS) have been the \textit{de facto} tools. Today, non-relational, cloud, or the so-called NoSQL databases are growing rapidly as an alternative model for database management. Often more features apply such as schema-free, easy replication support, simple API, eventually consistent / BASE (not ACID), a huge amount of data (big data) and more. \newline

In this work, we present an alternative, using one of the NoSQL databases (today, the volumes of data that can be handled by NoSQL systems exceed what can be handled by the biggest RDBMS) free available, to offer a different way of challenging these problems, always operating inside VO frame.
\chapter{Astronomy Big Data}


Big data are high-volume, high-velocity, and/or high-variety information assets that require new forms of processing to enable enhanced decision making, insight discovery and process optimization. Examples of big data include information such as Web search results, electronic messages (e.g., SMS, email and instant messages), social media postings, pictures, videos and even system log data.  However, it can also include credit/debit card transactions, check images, receipts and other transactional information depending on the source of the information. \\

Astronomical datasets are growing at an exponential rate: high performance computing applications in astronomy are enabling complex simulations with many billions of particles, while the forthcoming generation of telescopes will collect data at rates in excess of terabytes per day. This data deluge, both now and into the future, presents some critical challenges for the way astronomers derive new knowledge from their data.\\



In this chapter, we make an overview of some of the greatest telescopes, the way they acquire and store data and the problem they face.


\section{Atacama Large Millimiter Array}

\subsection{ALMA Instrument}

ALMA is a worldwide project; the synthesis of early visions of astronomers in its three partner communities: Europe, North America and Japan.\\

ALMA is a fusion of ideas, with its roots in three astronomical projects: the Millimeter Array (MMA) of the United States, the Large Southern Array (LSA) of Europe, and the Large Millimeter Array (LMA) of Japan.\\

\begin{figure}
\centering
\includegraphics[width=11cm,height=8cm]{images/alma.jpg}\\
\caption{ALMA dishes in Atacama Desert}
\end{figure}


\textbf{Millimeter Array} \\

The origins of the Millimeter Array (MMA) are found in the science of the NRAO 36-Foot Telescope (later known as the 12-Meter Telescope), soon followed by the 4.9m telescopes at the University of Texas and Aerospace Corporation, the 14m telescope at the Five Colleges Radio Astronomical Observatory, and the 7m telescope at AT\&T Bell Labs. The millimeter interferometers of the University of California (Berkeley) at the Hat Creek Observatory (later the Berkeley-Maryland- Illinois Association, or BIMA) and the California Institute of Technology at the Owens Valley Radio Observatory demonstrated the power that comes with high angular resolution for studying the sources found with the single dishes. The experience of using a powerful, flexible array that was provided by NRAO’s Very Large Array (VLA) at longer wavelengths was also very influential. Indeed, the prime characteristic of the MMA was the ability to obtain rapid high-quality images at 230GHz, that is, the MMA was to be a millimeter version of the VLA. The science targets of the MMA included the same broad range of topics seen at the VLA: (NSF) in July, 1990, called for an array of 40 antennas of 8-meter diameter, with four receiver bands covering the atmospheric windows from 30-350 GHz, configurable in four arrays of size 70-3000 m. The proposal discussed two possible sites for the MMA, both in the southwestern United States. Studies of the atmospheric transparency and phase stability at these sites led to similar studies on Mauna Kea, in Hawaii. Extensive atmospheric monitoring was also conducted there.\\


Concerns with the limited size of the area available to the MMA on Mauna Kea and with potential environmental problems prompted a search for potential sites in Chile. From April 1994, many highelevation sites were visited. Finally, the site retained for the MMA was formally proposed in 1996: it was the Chajnantor plateau. \\

This effort culminated in the signing of the ALMA Agreement on February 25, 2003, between the North American and European parties. More than 14 government agencies in Chile were involved in the negotiations.\\

Assuming all three partners are able to meet their commitments, it was decided that the final project would be cost-shared 37.5\% / 37.5\% / 25\% between North America, Europe, and Japan, respectively. The observing time, after a 10\% share for Chile, would be shared accordingly.\\

ALMA is an instrument that consist of a giant array of 12-m antennas with baselines up to 16 km, and an additional compact array of 7-m and 12-m antennas to greatly enhance ALMA's ability to image extended targets, located on the Chajnantor plateau at 5000m altitude. Initially, it will observe at wavelengths in the range 3 mm to 400 μm (84 to 720 GHz). The antennas can be moved around, in order to form arrays with different distributions of baseline lengths. More extended arrays will give high spatial resolution, more compact arrays give better sensitivity for extended sources. In addition to the array of 12-m antennas, there is the Atacama Compact Array (ACA), consisting of twelve 7-m antennas and four 12-m antennas. This array will mostly stay in a fixed configuration and is used to image large scale structures that are not well sampled by the ALMA 12-m array.\\

The design of ALMA is driven by three key science goals:\\

\begin{itemize}

\item The ability to detect spectral line emission from CO in a normal galaxy like the Milky Way at a redshift of $z=3$, in less than 24 hours

\item The ability to image the gas kinematics in protostars and in protoplanetary disks around young Sun-like stars in the nearest molecular clouds ($150 pc$)

\item The ability to provide precise high dynamic range images at an angular resolution of $0.1 arcsec$
\end{itemize}
 
ALMA delivers data cubes, of which the third axis is frequency. In this sense, the final data products are very much like that of an integral field unit with up to a million Spectral Pixels.\\

\textbf{Interferometry}\\
In order to obtain images, the raw visibility data need to be Fourier transformed. When ALMA is in full operations, this imaging step, as well as various calibration steps, will be done in the data reduction pipeline. Thus, fully calibrated data cubes will be delivered to the user. However, the imaging (and subsequent deconvolution) is a non-unique procedure, so users may want to redo these steps to optimize the data products for their scientific objectives. The Common Astronomy Software Applications package (CASA) has been developed for this purpose.\\

\textbf{Observing frequencies}\\
The frequency range available to ALMA is divided into different receiver bands. Data can only be taken in one band at a time. These bands range from band 3, starting at$ 84 GHz$, to band 10, ending at $~950 GHz$. For comparison, a frequency of $300 GHz$ translates to a wavelength of approximately $1mm$. Band 10 ($~869 GHz$) is planned. Two more bands (band 1 around $40 GHz$ and band 2 around $80 GHz$), might be added in the future. Initially, only six ALMA antennas will be equipped with band 5 receivers ($187 GHz$).\\

\textbf{Field of view}\\
The FWHM of the ALMA primary beam is 21" at $300 GHz$, and scales linearly with wavelength (diffraction limit of a single 12-m antenna, as opposed to that of the whole array). To achieve uniform sensitivity over a field larger than about a few arcsec, or to image larger regions than the primary beam, mosaicking is required, which is a standard observing mode for ALMA. If you plan to use mosaicking, individual pointings should be separated by 1/2 the primary beam FWHM to achieve Nyquist sampling.\\

\textbf{Spatial resolution}\\
The spatial resolution of ALMA depends on the observing frequency and the maximum baseline of the array, following the 1.2 x lambda/D scaling. In the most compact configurations ($~150 m$), resolutions range from 0.7" at $675 GHz$ to 4.8" at $110 GHz$. In the most extended configuration (~16 km in the completed array), the resolutions range from 6 mas at $675 GHz$ to 37 mas at $110 GHz$. These numbers refer to the FWHM of the synthesized beam (point spread function), which is the inverse Fourier transform of a (weighted) u-v sampling distribution. The resolution in arcsec can be approximated as: FWHM (") = 76 \/ max\_baseline (km) \/ frequency (GHz).\\

\textbf{Array configurations}\\
The ALMA 12-m array will cycle from its most compact configuration, with maximum baselines of ~150 m, to its most extended configuration, with maximum baselines of ~16 km (when completed), and back. The Atacama Compact Array (ACA) will have two configurations, one of which is a north-south extension to provide a better beam shape for far-north/far-south targets. Note that during the Early Science phase, the available configurations are restricted. See 'Capabilities' for conditions applying to the current Cycle.\\

\textbf{Spectral resolution}\\
ALMA can deliver data cubes with up to 7680 frequency channels (spectral resolution elements). The width of these channels can range between $3.8 kHz$ and $15.6 MHz$, but the total bandwidth cannot exceed $8 GHz$. At an observing frequency of $110 GHz$, the highest spectral resolution implies a velocity resolution of $0.01 km/s$, or $R=30,000,000$. At $110 GHz$, a velocity resolution of 1 km/s requires channel widths of $0.37 MHz$.\\

\textbf{Sensitivity}\\
For an interferometer, the noise level in the resulting data cubes (expressed in mJy) scales roughly as $S=(N*(Np*Δ\nu * Δ\tau)1/2)-1$, where N is the number of antennas, Np is the number of polarizations, Δν is the available bandwidth and Δτ is the observing time. For continuum observations, $Δ\nu=7.5 GHz$, for spectral line observations, Δν is the channel width. The ALMA Sensitivity Calculator can be used to estimate noise levels or required integration times to reach a desired noise level.

\subsection{ALMA Science Archive}

As stated in \cite{Etoka12}, the purpose of the ALMA archive is to provide services for:

\begin{itemize}
\item Persistent archiving and retrieval for observacional data.
\item Observaction descrpitors.
\item Datacubes produced by pipeline.
\item Technical and environmental data.
\end{itemize}

And the key-point of the conceptual design of the ALMA Archive is to guarantee that three ALMA Regional Centres (ARCs) hold an identical copy of the archive at the Joint ALMA Observatory (JAO) in Santiago. \\

Relational denormalized database.

ALMA frontend archive is optimized for storage and preservation, not for data query and retrieval. ASA database is inspired from ObsCore, RADAMS and Hubble Legacy Archive plus Virtual Observatory Software:

\begin{itemize}
\item openCADC, which is used for database access and VO access protocol
\item VOView, for Web components
\end{itemize}




\section{Square Kilometer Array}

Thousands of linked radio wave receptors will be located in Australia and in Southern Africa. Combining the signals from the antennas in each region will create a telescope with a collecting area equivalent to a dish with an area of about one square kilometre. \\

The Square Kilometer Array (SKA) will address fundamental unanswered questions about our Universe including how the first stars and galaxies formed after the Big Bang, how galaxies have evolved since then, the role of magnetism in the cosmos, the nature of gravity, and the search for life beyond Earth. \\

The SKA is a global science and engineering project led by the SKA Organisation, a not-for-profit company with its headquarters at Jodrell Bank Observatory, near Manchester, UK. \\

An array of dish receptors will extend into eight African countries from a central core region in the Karoo desert of South Africa. A further array of mid frequency aperture arrays will also be built in the Karoo. A smaller array of dish receptors and an array of low frequency aperture arrays will be located in the Murchison Radio-astronomy Observatory in Western Australia.\\

Construction is scheduled to start in 2016.\\

\begin{figure}
\centering
\includegraphics[width=11cm,height=8cm]{images/ska.jpg}\\
\caption{Artist's impression of the SKA dishes. Credit: SKA Organisation/TDP/DRAO/Swinburne Astronomy Productions}
\end{figure}


The recent launch of the Murchison Widefield Array (MWA) – a radio telescope based in Western Australia's Mid West - marked the start of an impressive flow of astronomical data that will be stored in the iVEC-managed Pawsey Center in Kensington for later use by researchers around the world.\\

According to Professor Andreas Wicenec, from The University of Western Australia node of the International Centre for Radio Astronomy Research (ICRAR), SKA has ``now have more than 400 megabytes per second of MWA data streaming along the National Broadband Network from the desert 800 km away''.\\

The Murchison Widefield Array is the first Square Kilometre Array precursor to enter full operations, generating a vast torrent of information that needs to be stored for later retrieval by researchers.\\

According to Proffesor Wicenec, ``To store the Big Data the MWA produces, you’d need almost three 1 TB hard drives every two hours`. The technical challenge isn’t just in saving the observations but how you then distribute them to astronomers from the MWA team in far-flung places so they can start using it''.\\

There are currently two links between the data stores in Perth and MWA researchers at the Massachusetts Institute of Technology (MIT) in the United States and the Victoria University of Wellington in New Zealand. A future link to India — another MWA partner — will also be created.\\

The data are not obviously intented to be fully available for everybody at everytime: for instance, MIT researchers are interested in the early universe so filtering techniques to control what data is copied from the Pawsey Center archive to the MIT machines are used. By 2013, more than 150 TB of data had been transferred automatically to the MIT store, with a stream of up to 4 TB a day increasing that value.\\

MWA is producing so much information that it would be impossible to manually decide where to send what, which is where a sophisticated archiving system — the open-source Next Generation Archive System (NGAS) — comes in. NGAS was initially developed by Professor Wicenec at the European Southern Observatory (ESO) and later modified by the ICRAR team to meet the MWA data challenge.\\

The NGAS operating mode is very simple, one simply asks the system for what he/she wants and it either provides it from the local store or retrieves it from the full archive back in Perth through a highly efficient dataflow management system.\\

About half of all MWA computing occurs on site in the Murchison, where signals from radio telescope antennas are combined and processed in a powerful system of computers called a correlator. What’s left to do in Perth is produce images, and manage storage and distribution by the archive system so MWA astronomers can analyze the collected data. Data travels down a dedicated 10 gigabit per second connection between the Murchison Radio-astronomy Observatory (MRO) and Geraldton, and the trip to Perth is completed on Australia’s new high-speed National Broadband Network.\\

The MWA will store about 3 Petabytes at the Pawsey Center each year. Another section of the Pawsey Center will be a supercomputing facility that includes computing for Australia’s other SKA precursor, CSIRO’s Australian Square Kilometre Array Pathfinder (ASKAP), and projects from geoscience and other computationally intensive fields.\\

To sum up, some relevants figures and facts about SKA:

\begin{itemize}

\item The data collected by the SKA in a single day would take nearly two million years to playback on an ipod.
\item The SKA central computer will have the processing power of about one hundred million PCs.
\item The SKA will use enough optical fibre to wrap twice around the Earth!
\item The dishes of the SKA will produce 10 times the global internet traffic.
\item The aperture arrays in the SKA could produce more than 100 times the global internet traffic.
\item The SKA will generate enough raw data to fill 15 million 64 GB iPods every day!
\item The SKA supercomputer will perform 1018 operations per second – equivalent to the number of stars in three million Milky Way galaxies – in order to process all the data that the SKA will produce.
\item The SKA will be so sensitive that it will be able to detect an airport radar on a planet 50 light years away.
\item The SKA will contain thousands of antennas with a combined collecting area of about one square kilometre (that’s 1 000 000 square metres!).
\item Analysts estimate the London Olympics was the most data-heavy yet – with some 60 Gbytes, the equivalent of 3,000 photographs, travelling across the network in the Olympic Park every second. This however is only equivalent to the data rate from about half a low frequency aperture array station in SKA phase one.

\end{itemize}
\chapter{The Virtual Observatory} % (fold)
\label{cha:the_virtual_observatory}


The International Virtual Observatory Alliance\urlnote{http://www.ivoa.net/}
(IVOA) was formed in 2002 with the aim to
\emph{``facilitate the international coordination and collaboration necessary for the development and deployment of the tools, systems and organizational structures necessary to enable the international utilization of astronomical archives as an integrated and interoperating virtual observatory.''}
The IVOA now comprises programs from several countries and intergovernmental organizations like ESA and ESO. 

The IVOA focuses on the development of standards and encourages their implementation for the benefit of the worldwide astronomical community.
As a result, a federation of lightly-coupled, interoperable web-services provide astronomers with data from multiple data archives and catalogues.
Working Groups are constituted with cross-program membership in those areas where key interoperability standards and technologies have to be defined and agreed upon. The Working Groups develop standards using a process modeled after the World Wide Web Consortium, in which Working Drafts progress to Proposed Recommendations and finally to Recommendations. Recommendations may ultimately be endorsed by the Virtual Observatory Working Group of Commission 5 (Astronomical Data) of the International Astronomical Union. The IVOA also has Interest Groups that discuss experiences using VO technologies and provide feedback to the Working Groups. 

In this section we are not going to make a deep study of the Virtual Observatory techniques, technologies, protocols and interfaces, just those needed and selected to explain our proposal of including NoSQL into VO. Bearing in mind that the problem we are trying to address is data modelling, we will just make an overview of those aspects related with our work.

\section{Flexible Image Transport System} % (fold)
\label{sec:flexible_image_transport_system}


The
Flexible Image Transport System (FITS) is an open standard defining a digital file format useful for storage, transmission and processing of scientific and other images. FITS is the most commonly used digital file format in astronomy. Unlike many image formats, FITS is designed specifically for scientific data and hence includes many provisions for describing photometric and spatial calibration information, together with image origin metadata. The FITS format was first standardized in 1981; it has evolved gradually since then, and the most recent version (3.0) was standardized in 2008. 
  
FITS is also often used to store non-image data, such as spectra, photon lists, data cubes, or even structured data such as multi-table databases. A FITS file may contain several extensions, and each of these may contain a data object. For example, it
could be % is
possible to store
X-ray % x-ray
and infrared exposures in the same file. 
 
FITS support is available in a variety of programming languages that are used for scientific work, including C, C++, C\#, Fortran, IDL, Java, Mathematica, MatLab, Perl, PDL, or Python. 
 
Image processing programs such as GIMP can generally read simple FITS images, but cannot usually interpret complex tables and databases.

\subsection{FITS Data Format}

Each FITS file consists of one or more headers containing ASCII card images that carry keyword/value pairs, interleaved between data blocks. The keyword/value pairs provide information such as size, origin, coordinates, binary data format, free-form comments, history of the data, and anything else the creator desires. In more technical terms, a FITS file is comprised of parts called Header Data Units (HDU), being the first HDU called primary HDU o primary array. This array can contain a 1-999 dimensional array. A typical primary array could contain a 1D spectrum, 2D image or 3D data cube. Any number of HDU can follow the main array, and are called FITS extensions. Currently, three different extensions can be defined:

\begin{itemize}
\item Image extension, a 0-9999 dimensional array of pixels, which begins with \texttt{XTENSION = `IMAGE'}
\item ASCII table extension which stores tabular data in ASCII formats. They begin with \texttt{XTENSION = `TABLE'}
\item Binary table extension stores tabular data in binary representation. Headers start with \texttt{XTENSION = `BINTABLE'}
\end{itemize}

Besides, there are additional type of HDU called random groups, but only used for radio interferometry.
      
\begin{figure}[tb]
\centering
\includegraphics[width=11cm,height=8cm]{images/fits_header.png}
\caption{Viewing a FITS header in Aladin}
\end{figure}

% section flexible_image_transport_system (end)

\section{Table Access Protocol} % (fold)
\label{sec:table_access_protocol}

The Table Access Protocol~\cite{2010tap..irec.....D} (TAP) defines a web service for accessing tables containing astronomical catalogues. TAP is the protocol which underlies in the process of posing a query against a data source (or several data sources). The result of a query is a table, usually a VOTable.\footnote{Support for VOTable output is mandatory, while other formats may be available.}  

Queries which use TAP protocol can be made through several clients, like:
\begin{itemize}
\item TOPCAT\urlnote{http://www.star.bris.ac.uk/~mbt/topcat/}
\item TAPHandle\urlnote{http://saada.unistra.fr/taphandle/}
is a TAP client
 which operates fully within the Web browser
\item The TAP shell\urlnote{http://vo.ari.uni-heidelberg.de/soft/tapsh}, a command line interface to querying TAP servers, complete with metadata management and command line completion.
\item The GAVO VOTable library\urlnote{http://docs.g-vo.org/DaCHS/votable.html}, which allows embedding TAP queries in Python.
\end{itemize}

The types of queries implemented by TAP are:

\begin{itemize}
\item data queries, where the query result is directly the science data to be operated upon;
\item metadata queries, where the query result is metadata of the datasets, the archival service, or both; and
\item Virtual Observatory Support Interface (VOSI) queries, which provide information on the different TAP service capabilities.
\end{itemize}

TAP includes support for multiple query languages, including queries specified using the Astronomical Data Query Language (ADQL; see~\cite{2008adql.ivoav0910O}) and the Parameterised Query Language~\cite{IVOA-Data-Access-Layer-Working-Group:2009lr} (PQL, under development). Other query languages are also supported, and this mechanism allows developments outside the IVOA to be used without modifying the TAP specification. Finally, it also includes support for both synchronous and asynchronous queries. Special support is provided for spatially indexed queries using the spatial extensions in ADQL. 

Listing~\ref{lst:adqlsample} shows an ADQL sample query: a crossmatch between a user-uploaded catalog (\texttt{TAP\_UPLOAD.T1}), and
the Third US Naval Observatory CCD Astrograph Catalog\urlnote{http://www.usno.navy.mil/USNO/astrometry/optical-IR-prod/ucac} (UCAC3) % UCAC
catalog, compensating the fit for proper motion.

\begin{lstlisting}[float,language=SQL,caption={ADQL sample query: the results will be a crossmatch between a user-uploaded catalog, and the UCAC3 catalog, compensating the fit for proper motion.},label=lst:adqlsample]
SELECT 
 u.raj2000+d_alpha+d_pmalpha/cos(radians(u.dej2000))*(u.epoch-2000) AS ra_icrs,
 u.dej2000+d_delta+d_pmdelta*(u.epoch-2000) AS de_cicrs,
 u.pmra+d_pmalpha AS pmra_icrs,
 u.pmde+d_pmdelta AS pmde_icrs,
 u.*
FROM
  TAP_UPLOAD.T1 AS u
  JOIN ucac3.icrscorr AS c
  ON (c.alpha=FLOOR(u.raj2000)+0.5 and c.delta=FLOOR(u.dej2000)+0.5)
\end{lstlisting}

\subsection{ObsCore and ObsTAP} % (fold)
\label{sub:obstap}

A data model is a description of the objects represented by a computer system with their properties and relationships, a logical model detailing the decomposition of a complex dataset into simpler elements, like a collection of concepts and rules used in defining data model. 

In 2011 IVOA proposed a new recommendation: Observation Data Model Core
Components~\cite{2011arXiv1111.1758L} (ObsCore). ObsCore is a 
generic data model for the metadata necessary to describe any astronomical observation. Practically all of the ObsCore metadata (which is modelled as a single table with multiple columns) can be found in the FITS headers of reduced datasets.

The ObsCore document was intended not just to be a description of said data model, but as an interface specification which integrated the data modeling and data access aspects in a single service, called ObsTAP.
In other words, ObsTAP is the combination of the TAP with the ObsCore data model.

% subsection obstap (end)

\subsection{Universal Worker Service} % (fold)
\label{sub:universal_worker_service}

The Universal Worker Service~\cite{2010uws..irec.....H} (UWS) 
standard defines a work
pattern 
that specifies
how to build asynchronous (as the client does not wait for each request to be fulfilled; even if the client disconnects from the service, activity is not aborted), stateful (the service remembers results of a previous activity), job-oriented
services\footnote{Job specification rules following the UWS pattern are written in the Job Description Language (JDL).}.

When the execution starts, the job can adopt several states, following the state machine illustrated in Fig.~\ref{fig:jobstatuses}.

\begin{figure}[tb]
\centering
\includegraphics[height=8cm]{images/UWSStates.png}
\caption{Universal Worker Server Job states}
\label{fig:jobstatuses}
\end{figure}

% subsection universal_worker_service (end)

% section table_access_protocol (end)

\section{OpenCADC} % (fold)
\label{sec:opencadc}

The
OpenCADC\urlnote{https://code.google.com/p/opencadc/source/browse}
is a software framework, developed by the
Canadian Astronomy Data Center (CADC), and which implements a number of VO standards. It can be used either directly or as a library to implement VO services.

OpenCADC is used for implementing the query capabilities of the ASA (see Fig.~\ref{fig:asa_implementation}). It is composed of several libraries, but the most important from the point of view of their interface are cadcUWS, which implements the UWS pattern, and cadcTAP, which implements the TAP web interface.

\subsection{cadcUWS library} % (fold)
\label{sub:cadcuws_library}

We will briefly describe
the structure of the code relating UWS in OpenCADC (the cadcUWS Library), the section of the code provides Job class and plugin architecture, servlet with UWS async and sync behavior. These classes are illustrated in Fig.~\ref{fig:uwsjobs}, and listed below.

\begin{figure}[tb]
\centering
\includegraphics[height=8cm]{images/Class_Diagram__UWS__UWSObjects.png}
\caption{Class diagram representing UWS objects}
\label{fig:uwsjobs}
\end{figure}

\begin{description}

\item[JobManager] Class responsible for job control.

\item[JobPersistence] Class in In charge of storing and retrieving Job state.

\item[JobExecutor] Class which executes every job in separated threads.

\item[JobRunner] The code that actually executes the job.

\end{description}

% subsection cadcuws_library (end)

\subsection{cadcTAP library} % (fold)
\label{sub:cadctap_library}

The cadcTAP library is responsible of async and sync queries, where QueryRunner implements JobRunner. It also contains \texttt{TAP\_SCHEMA}\footnote{TAP services try to be self-describing about what data they contain. They provide information on what tables they contain in special tables in \texttt{TAP\_SCHEMA}}
Data Definition Language\footnote{In RDBMS, the Data Definition Language is the language that allows for the creation of tables, fields, and data types.} (DDL) % DDL
statements and is used by query parser to validate table and column usage.

\begin{description}
	\item[TapQuery Interface] It has a separate implementation for each \texttt{LANG} (e.g. \texttt{LANG=ADQL}) specified and processes the query to local SQL.
	\item[SqlQuery] When code states \texttt{LANG=SQL}, it implements TapQuery and fully navigates it (\texttt{FROM}, \texttt{WHERE} and \texttt{HAVING} clauses).
	
	\item[AdqlQuery] Same as before when \texttt{LANG=ADQL}.
	
	
	\item[Plugins] The QueryRunner class (below) needs to be able to find the following classes:

	\begin{description}
	\item[UploadManager] Interface for classes responsible for uploading local data to the remote TAP service.
	\item[TableWriter] Interface for classes that convert the retrieved data into a particular data format.
	\item[FileStore] Interface for classes that deal with the local file storage.
	\end{description}
	
	\item[QueryRunner] Class that implements JobRunner and sets Job state, finds a DataSource and uses TapSchema, UploadManager, TapQuery and TableWriter.
	
\end{description}

% subsection cadctap_library (end)



% section opencadc (end)



% chapter the_virtual_observatory (end)

\chapter{Relational DBMS vs Document oriented DBMS}

Typically, modern relational databases have shown little efficiency in certain applications using intensive data, like indexing of a large number of documents, sites rendering with high traffic, and streaming sites. Typical RDBMS implementations are tuned either for small but frequent reads and writes or a large set of transactions that have few write accesses. On the other hand NoSQL can serve load lots of reads and writes.\\

We will start making a short overview of the relational model using Postgresql RDMS. Then we move to the NoSQL alternative, through Mongo database.\\

\section{Relational Database Management System}

\subsection{Postgresql}


PostgreSQL \footnote{We are not going to explain relational model theory in this section. We will focus on one of the most used and open DBMS -Posgresql-. For a full description of relational model, there are several sources available:
\url{http://infolab.stanford.edu/~ullman/focs/ch08.pdf},
\url{http://www.macs.hw.ac.uk/~trinder/DbInfSystems/3-RelModel.pdf} or \cite{Silberschatz_01}}
 is a powerful, open source object-relational database system. It has more than 15 years of active development and a proven architecture that has earned it a strong reputation for reliability, data integrity, and correctness. It runs on all major operating systems. It is fully ACID compliant, has full support for foreign keys, joins, views, triggers, and stored procedures (in multiple languages). It has native programming interfaces for C/C++, Java, .Net, Perl, Python, Ruby, Tcl, ODBC, among others. \\

An enterprise class database, PostgreSQL boasts sophisticated features such as Multi-Version Concurrency Control (MVCC), point in time recovery, tablespaces, asynchronous replication, nested transactions (savepoints), online/hot backups, a sophisticated query planner/optimizer, and write ahead logging for fault tolerance. It is highly scalable both in the sheer quantity of data it can manage and in the number of concurrent users it can accommodate. There are active PostgreSQL systems in production environments that manage in excess of 4 terabytes of data.\\

PostgreSQL runs stored procedures in more than a dozen programming languages, including Java, Perl, Python, Ruby, Tcl, C/C++, and its own PL/pgSQL, which is similar to Oracle's PL/SQL. Included with its standard function library are hundreds of built-in functions that range from basic math and string operations to cryptography and Oracle compatibility. Triggers and stored procedures can be written in C and loaded into the database as a library, allowing great flexibility in extending its capabilities.\\

Just as there are many procedure languages supported by PostgreSQL, there are also many library interfaces as well, allowing various languages both compiled and interpreted to interface with PostgreSQL. There are interfaces for Java (JDBC), ODBC, Perl, Python, Ruby, C, C++, PHP, Lisp, Scheme, and Qt just to name a few.\\

PostgreSQL's source code is available under a liberal open source license: the PostgreSQL License. This license gives you the freedom to use, modify and distribute PostgreSQL in any form you like, open or closed source. Any modifications, enhancements, or changes you make are yours to do with as you please.\\

\begin{table}
\begin{center}
\begin{tabular}{|l|l|}
\hline \textbf{Limit} & \textbf{Value}\\ \hline
Maximum Database Size & Unlimited\\
Maximum Table Size & 32 TB\\
Maximum Row Size & 1.6 TB\\
Maximum Field Size & 1 GB\\
Maximum Rows per Table & Unlimited\\
Maximum Columns per Table & 250 - 1600 depending on column types\\
Maximum Indexes per Table & Unlimited\\ \hline
\end{tabular}
\end{center}
\caption{PostgreSQL DBMS Limits}
\end{table}


\section{NoSQL}

NoSQL implementations used in the real world include 3TB Digg green markers (indicated to highlight the stories voted by others in the social network), the 6 TB of "ENSEMBLE" European Commission database used in comparing models and air quality, and the 50 TB of Facebook inbox search.\\

NoSQL architectures often provide limited consistency, such as events or transactional consistency restricted to only data items. Some systems, however, provide all guarantees offered by ACID systems by adding an intermediate layer. There are two systems that have been deployed and provide for storage of snapshot isolation column: Google Percolator (based on BigTable system) and Hbase transactional system developed by the University of Waterloo. These systems use similar concepts in order to achieve distributed multiple rows ACID transactions with snapshot isolation guarantees for the underlying storage system in that column, wit no extra overhead in data management, no system deployment middleware or any maintenance introduced by middleware layer.\\

Quite NoSQL systems employ a distributed architecture, maintaining data redundantly on multiple servers, often using distributed hash table. Thus, the system may actually escale adding more servers, and thus a server failure may be tolerated.\\

There are different NoSQL DBs for different projects:

\begin{itemize}

\item Document oriented

  \begin{itemize}
    \item CouchDB
    \item MongoDB
    \item RavenDB
    \item IBM Lotus Domino
  \end{itemize}

\item Graph oriented

  \begin{itemize}
    \item Neo4j
    \item AllegroGraph
    \item InfiniteGraph
    \item Sones GraphDB
    \item HyperGraphDB
  \end{itemize}

\item Key-value oriented

  \begin{itemize}
    \item Cassandra
    \item BigTable
    \item Dynamo (Amazon)
    \item MongoDB
    \item Project Voldemort (LinkedIn)
  \end{itemize}

\item Multivalue

  \begin{itemize}
    \item OpenQM
  \end{itemize}  

\item Object Oriented
  
  \begin{itemize}
    \item Zope Object Database
    \item db4o
    \item GemStone S
    \item Objectivity/DB
  \end{itemize}

\item Tabular

  \begin{itemize}
    \item HBase
    \item BigTable
    \item LevelDB (BigTable open version)
    \item Hypertable
  \end{itemize}
  

\end{itemize}

They run on clusters of inexpensive machines.\\



\subsection{MongoDB: a document oriented database}

MongoDB (from "humongous") is an open source document-oriented database system developed and supported by 10gen. It is part of the NoSQL family of database systems. Instead of storing data in tables as is done in a "classical" relational database, MongoDB stores structured data as JSON-like documents with dynamic schemas (MongoDB calls the format BSON), making the integration of data in certain types of applications easier and faster. \\

10gen began Development of MongoDB in October 2007 and was not created to be just another database that tries to do everything for everyone. Instead, MongoDB was created to work with documents rather than rows, was extremely fast, massively scalable, and easy to use. In order to accomplish this, some features were excluded, namely support for transactions.

%\begin{shaded}
A document database is more like a collection of documents. Each entry is a document, and each one can have its own structure. If you want to add a field to an entry, you can do so without affecting any other entry.
%\end{shaded} 

\subsubsection{Main features}

\begin{itemize}

\item \textbf{Ad hoc queries} \\
MongoDB supports search by field, range queries, regular expression searches. Queries can return specific fields of documents and also include user-defined JavaScript functions.

\item \textbf{Indexing} \\
Any field in a MongoDB document can be indexed (indices in MongoDB are conceptually similar to those in RDBMSes). Secondary indices are also available.

\item \textbf{Replication} \\
MongoDB supports master-slave replication. A master can perform reads and writes. A slave copies data from the master and can only be used for reads or backup (not writes). The slaves have the ability to select a new master if the current one goes down.

\item \textbf{Load balancing} \\
MongoDB scales horizontally using sharding.[9] The developer chooses a shard key, which determines how the data in a collection will be distributed. The data is split into ranges (based on the shard key) and distributed across multiple shards. (A shard is a master with one or more slaves.)
MongoDB can run over multiple servers, balancing the load and/or duplicating data to keep the system up and running in case of hardware failure. Automatic configuration is easy to deploy and new machines can be added to a running database.

\item \textbf{File storage} \\
MongoDB could be used as a file system, taking advantage of load balancing and data replication features over multiple machines for storing files.
This function, called GridFS,[10] is included with MongoDB drivers and available with no difficulty for development languages (see "Language Support" for a list of supported languages). MongoDB exposes functions for file manipulation and content to developers. GridFS is used, for example, in plugins for NGINX and lighttpd. In a multi-machine MongoDB system, files can be distributed and copied multiple times between machines transparently, thus effectively creating a load balanced and fault tolerant system.

\item \textbf{Aggregation} \\
MapReduce can be used for batch processing of data and aggregation operations. The aggregation framework enables users to obtain the kind of results for which the SQL GROUP BY clause is used.

\item \textbf{Server-side JavaScript execution} \\
JavaScript can be used in queries, aggregation functions (such as MapReduce), are sent directly to the database to be executed.

\item \textbf{Capped collections} \\
MongoDB supports fixed-size collections called capped collections. This type of collection maintains insertion order and, once the specified size has been reached, behaves like a circular queue.

Once we have seen the main features of MongoDB, we can move on to the language itself.
\end{itemize}


\subsubsection{The basics}

We must know four concepts to dig into MongoDB's world:

\begin{itemize}
\item Database: this concept is much like the RDBM counterpart.
\item Collection: we can see a collection and a table as the same thing.
\item Document: its equivalent in RDBM is the row, and a document is made up of fields.
\item Field: is a lot like a column.
\end{itemize}


\begin{figure}[H]
\centering
\includegraphics[width=11cm,height=8cm]{images/mongo_dia.png}\\
\caption{Tree View for Tap Schema in MongoVue}
\end{figure}


\subsection{Advantages and uncertainties of using a NoSQL solution}

\begin{itemize}

\item \textbf{Elastic scaling}

For years, database administrators have relied on scale up — buying bigger servers as database load increases — rather than scale out — distributing the database across multiple hosts as load increases. However, as transaction rates and availability requirements increase, and as databases move into the cloud or onto virtualized environments, the economic advantages of scaling out on commodity hardware become irresistible.

RDBMS might not scale out easily on commodity clusters, but the new breed of NoSQL databases are designed to expand transparently to take advantage of new nodes, and they’re usually designed with low-cost commodity hardware in mind.


\item \textbf{Big data}

Just as transaction rates have grown out of recognition over the last decade, the volumes of data that are being stored also have increased massively. O’Reilly has cleverly called this the “industrial revolution of data.” RDBMS capacity has been growing to match these increases, but as with transaction rates, the constraints of data volumes that can be practically managed by a single RDBMS are becoming intolerable for some enterprises. Today, the volumes of “big data” that can be handled by NoSQL systems, such as Hadoop, outstrip what can be handled by the biggest RDBMS.


\item \textbf{No need for DBAs}

Despite the many manageability improvements claimed by RDBMS vendors over the years, high-end RDBMS systems can be maintained only with the assistance of expensive, highly trained DBAs. DBAs are intimately involved in the design, installation, and ongoing tuning of high-end RDBMS systems.

NoSQL databases are generally designed from the ground up to require less management:  automatic repair, data distribution, and simpler data models lead to lower administration and tuning requirements — in theory. In practice, it’s likely that rumors of the DBA’s death have been slightly exaggerated. Someone will always be accountable for the performance and availability of any mission-critical data store.


\item \textbf{Economics}

NoSQL databases typically use clusters of cheap commodity servers to manage the exploding data, while RDBMS tends to rely on expensive proprietary servers and storage systems. The result is that the cost per gigabyte or transaction/second for NoSQL can be many times less than the cost for RDBMS, allowing you to store and process more data at a much lower price point.


\item \textbf{Flexible data models}

Change management is a big headache for large production RDBMS. Even minor changes to the data model of an RDBMS have to be carefully managed and may necessitate downtime or reduced service levels.

NoSQL databases have far more relaxed — or even nonexistent — data model restrictions. NoSQL Key Value stores and document databases allow the application to store virtually any structure it wants in a data element. Even the more rigidly defined BigTable-based NoSQL databases (Cassandra, HBase) typically allow new columns to be created without too much fuss.

The result is that application changes and database schema changes do not have to be managed as one complicated change unit. In theory, this will allow applications to iterate faster, though,clearly, there can be undesirable side effects if the application fails to manage data integrity.

\end{itemize}


NoSQL systems have generated a lot of enthusiasm but there are still a lot of questions about its future:

\begin{itemize}

\item \textbf{Maturity}

RDBMS systems have been around for a long time. NoSQL advocates will argue that their advancing age is a sign of their obsolescence, but for most CIOs, the maturity of the RDBMS is reassuring. For the most part, RDBMS systems are stable and richly functional. In comparison, most NoSQL alternatives are in pre-production versions with many key features yet to be implemented. Living on the technological leading edge is an exciting prospect for many developers, but enterprises should approach it with extreme caution.

\item \textbf{Support}

Enterprises want the reassurance that if a key system fails, they will be able to get timely and competent support. All RDBMS vendors go to great lengths to provide a high level of enterprise support.

In contrast, most NoSQL systems are open source projects, and although there are usually one or more firms offering support for each NoSQL database, these companies often are small start-ups without the global reach, support resources, or credibility of an Oracle, Microsoft, or IBM.
 
\item \textbf{Analytics and business intelligence}

NoSQL databases have evolved to meet the scaling demands of modern Web 2.0 applications. Consequently, most of their feature set is oriented toward the demands of these applications. However, data in an application has value to the business that goes beyond the insert-read-update-delete cycle of a typical Web application. Businesses mine information in corporate databases to improve their efficiency and competitiveness, and business intelligence (BI) is a key IT issue for all medium to large companies.

NoSQL databases offer few facilities for ad-hoc query and analysis. Even a simple query requires significant programming expertise, and commonly used BI tools do not provide connectivity to NoSQL.

Some relief is provided by the emergence of solutions such as HIVE or PIG, which can provide easier access to data held in Hadoop clusters and perhaps eventually, other NoSQL databases. Quest Software has developed a product — Toad for Cloud Databases — that can provide ad-hoc query capabilities to a variety of NoSQL databases.

\item \textbf{Administration}

The design goals for NoSQL may be to provide a zero-admin solution, but the current reality falls well short of that goal. NoSQL today requires a lot of skill to install and a lot of effort to maintain.

\item \textbf{Expertise}

There are literally millions of developers throughout the world, and in every business segment, who are familiar with RDBMS concepts and programming. In contrast, almost every NoSQL developer is in a learning mode. This situation will address naturally over time, but for now, it is by far easier to find experienced RDBMS programmers or administrators than NoSQL experts.

\end{itemize}


\section{A FITS alternative with MongoDB}
\chapter{Rewriting ASA with MongoDB}

Esto es una prueba
\chapter{Successful case studies}

\section{CMS at the LHC}

High-energy physicists working at the Compact Muon Solenoid (CMS) detector at the Large Hadron Collider (LHC) at Cern in Switzerland are benefiting from a NoSQL database management system that gives them unified access to data from a slew of sources.

Valentin Kuznetsov, a research associate and computer specialist at Cornell University, is a member of a team providing data management to the CMS Cern project. It built a system using MongoDB in preference to relational database technologies and other non-relational options.

``We considered a number of different options, including file-based and in-memory caches, as well as key-value databases, but ultimately decided that a document-oriented database would best suit our needs," he says. "After evaluating several applications, we chose MongoDB due to its support of dynamic queries, full indexes, including inner objects and embedded arrays, as well as auto-sharding.''

The CMS (pictured) is one of two particle physics detectors at Cern. It collects data from the LHC experiment that reproduces the big bang that kick-started the universe, designed to gain an understanding of how matter and force particles get their mass. More than 3,000 physicists from 183 institutions representing 38 countries are involved in the design, construction and maintenance of the CMS experiments.

Cornell is one of many institutions worldwide that contribute to the LHC experiments at Cern. Kuznetsov, a physicist and software engineer who has worked at Cern in the past, is involved in the data management group at Cern.

He says some five years ago the data management group at the CMS confronted a data discovery problem, with a variety of databases necessitating a user interface that would hide the complexity of the underlying architecture from the physicists. It wanted to build a Google-like interface, but one that would return precise answers to queries whose form could not be determined in advance.

``We had a variety of distributed databases and different formats – HTML, XML, JSON files, and so on. And then there is the security dimension. The complexity was, and is, huge'', he says.

The team discovered the world of NoSQL databases and, within that, the document-oriented approach seemed the best fit, with MongoDB being favoured because it lent itself well to the end product being an interface that could be queried in free text form.

MongoDB is one of a sub-genre of document store NoSQL databases, like Apache’s CouchDB. 

Barry Devlin, a data warehousing expert who blogs on TechTarget’s B-Eye Network, explains what is meant by a document-oriented approach: ``Documents? If you're confused, you are not alone. In this context, we don't mean textual documents used by humans, but rather use the word in a programming sense as a collection of data items stored together for convenience and ease of processing, generally without a pre-defined schema.''

The CMS spans more than 100 datacentres in a three-tier model and generates around 10PB of data each year in real data, simulated data and metadata. This information is stored and retrieved from relational and non-relational data sources, such as relational databases, document-oriented databases, blogs, wikis, file systems and customised applications.

To provide the ability to search and aggregate information across this complex data landscape CMS's Data Management and Workflow Management (DMWM) project created a data aggregation system (DAS), built on MongoDB. 

The DAS provides a layer on top of the existing data sources that allows researchers and other staff to query data via free text-based queries, and then aggregates the results from across distributed providers, while preserving their integrity, security policy and data formats. The DAS then represents that data in defined format.

All DAS queries can be expressed in a free text-based form, either as a set of keywords or key-value pairs, where a pair can represent a condition. Users can query the system using a simple, SQL-like language, which is then transformed into the MongoDB query syntax, which is itself a JSON (JavaScript Object Notation) record, said Kuznetsov.

``Due to the schema-less nature of the underlying MongoDB back end we are able to store DAS records of any arbitrary structure, regardless of whether it's a dictionary, lists, key-value pairs, and so on. Therefore, every DAS key has a set of attributes describing its JSON structure,'' says Kuznetsov.

Kuznetsov says it is now clear that ``there was nothing specific to the system related to our experiment''. The approach is extensible. 

At Cornell, other groups, including in ornithology, are facing similar problems and have expressed interest. He stresses that “analytics should be the main ingredient. The system should learn from what is being asked and be able to answer more questions in a free way.

``The beauty of MongoDB,`` he says, ``is that the query language is built into the system,`` unlike, for example, Couch, also in use at Cern, where users need to code for each query.

''DAS is used 24 hours a day, seven days a week, by CMS physicists, data operators and data managers at research facilities around the world. The average query may resolve into thousands of documents, each a few kilobytes in size. [We get a] throughput of around 6,000 documents a second for raw cache population,`` concludes Kuznetsov.

The Cern physicists are at work unlocking the mysteries of the universe, and NoSQL technology is, whether they know that or not, helping them do it.


%%%%%

\section{PanDA Workload Management System}

PanDA is a Workload Management System built around the concept of Pilot Frameworks [1]. In this approach, workload is assigned to successfully activated and validated Pilot Jobs, which are lightweight processes that probe the environment and act as a ‘smart wrapper’ for the payload. This 'late binding' of workload jobs to processing slots prevents latencies and failures in slot acquisition from impacting the jobs, and maximizes the flexibility of job allocation to globally distributed and heterogeneous resources [2]. The system was developed in response to computing demands of the ATLAS collaboration at the LHC, and for a number of years now has been the backbone of ATLAS data production and user analysis.

A central and crucial component of the PanDA architecture is its database (hosted on an Oracle RDBMS), which at any given time reflects the state of both pilot and payload jobs, as well as stores a variety of vital configuration information. It is driven by the PanDA Server, which is implemented as an Apache-based Python application and performs a variety of brokerage and workload management tasks, as well as advanced pre-emptive data placement at sites, as required by the workload.

By accessing the central database, another PanDA component -the PanDA Monitoring System (or simply PanDA Monitor)- offers to its users and operators a comprehensive and coherent view of the system and job execution, from high level summaries to detailed drill-down job diagnostics. Some of the entities that are monitored in the system are:

Pilot Job Generators (schedulers)
Queues (abstractions of physical sites)
Pilot Jobs (or simply pilots)
Payload Jobs (the actual workload)

All of these are related to each other by reference, i.e. an instance of the Pilot Job Generator is submitting pilots to a particular queue, a queue in turn can contain any number of pilots in various stages of execution, and payload jobs are mapped to successfully activated, running pilots. Payload jobs are also associated with their input/output datasets (details of data handling in PanDA go beyond the scope of this paper). The purpose of the PanDA Monitoring System is to present these objects and their logical associations to the user with optimal utility.

Characteristics of the data and current design of the PanDA monitor
As already mentioned, the PanDA Monitor handles a variety of data, and perhaps the most important and widely accessed objects stored in its database are entries describing payload jobs. We used this type of data as the most relevant case study in considering our database strategies going forward. Each job entry is mapped onto a row in the Oracle table, with roughly a hundred attributes (columns).

The monitor is currently implemented as a Python application embedded in the Apache server. Database access is done via the cx\_Oracle Python library. The Oracle server is handling loads of the order of 103 queries of varying complexity per second. There are more than 500,000 new entries generated daily, which results in multi-terabyte accumulated data load. Despite ongoing query optimization effort, there are capability and performance limits imposed by available resources of the current Oracle installation and nature of the queries themselves.

Since RDBMS features such as the ability to perform join operation across several tables can be costly in terms of performance, most of the widely accessed data stored in the PanDA Oracle database is de-normalized. For that reason, there is slight redundancy in the data and joins are not used in the application.

For performance and scalability reasons, the data stored in Oracle is partitioned into ''live'' and ''archive'' tables. The former contains the state of live objects (such as job being queued or executed in one of the clouds), while the latter contains a much larger set of data which is final (read-only, such as parameters of fully completed jobs) and has been migrated from the live table after a certain period of time after becoming static. In addition, we note that monitoring applications do not require guaranteed “hard consistency” of the data, which would be another crucial motivation to use a RDBMS. In other words, the time window between the instants when information is updated in the database by its client application, and when this is reflected in results of subsequent queries, is not required to be zero (which is achieved in RDBMS by making the “write” operation synchronous).
Motivation for noSQL and choice of platform
As explained above, we actually don’t have compelling reasons to store the monitoring data in a RDBMS. In situations like this, a variety of so-called noSQL database solutions are often employed by major Web services that require extreme degrees of availability, resilience and scalability. Examples include Google, Amazon, Facebook, Twitter, Digg and many others, as well as “brick and mortar” businesses. In order to leverage this new technology, we initiated a R\&D project aiming at applying this approach to elements of the PanDA monitoring system to make it future-proof and gain relevant experience going forward.
The “noSQL” term applies to a broad class of database management systems that differ from RDBMS in some significant aspects. These systems include such diverse categories as Document Store, Graph Databases, Tabular, Key-Value Store and others. Based on the patterns observed in queries in PanDA, a conclusion was made that either a Key-Value or a Tabular Store would be an adequate architecture in the context of PanDA monitoring system. A significant portion of all queries done in PanDA is performed using a unique identifier each entry has, therefore representing a classic case of Key-Value query. Other types of queries are done by indexing relevant columns, which is a general approach that will still work with noSQL solutions which support indexing.
We considered two of the most popular platforms: Apache Cassandra (Key-Value) and Apache Hbase (Tabular). The latter relies on Hadoop Distributed Filesystem. To cut on the learning curve, and for ease of operation, it was decided to choose Cassandra, which does not have such dependency. In addition, Cassandra was already deployed and undergoing evaluation by ATLAS personnel, giving us leverage in quickly acquiring our own R\&D platform.
Cassandra characteristics
A quick description of Cassandra would be as follows: it’s a truly distributed, schema-less, eventually consistent, highly scalable key-value storage system.
The key, in this context, can be of almost any data type, as Cassandra handles it as an array of bytes. The value is a collection of elements called columns (again, stored as arrays of bytes). The column is the smallest unit of data in Cassandra and consists of a name, value and a timestamp (used by Cassandra internally). This gives the developer flexibility in modeling objects in the data. Aggregation of columns with the same key is often called a row, and it can be thought of as a dictionary describing an object. A collection of rows pertaining to the same application is termed a column family and serves as a very distant approximation of what is called a table in the traditional RDBMS – note however that Cassandra rows can be drastically different from each other inside a single column family, while rows in a RDBS table follow the same schema.
 By design, the write operation is durable: once a write is completed (and this becomes known to the client requesting this procedure), data will survive many modes of hardware failure. While an instance of Cassandra can be successfully run on a single node, its advantages are best utilized when using a cluster of nodes, possibly distributed across more than one data center, which brings the following features:
Incremental and linear scalability – capacity can be added with no downtime
Automatic load balancing
Fault tolerance, with data replication within the cluster as well as across data centers

Cassandra also features built-in caching capabilities which will not be discussed here for sake of brevity. The application is written in Java, which makes it highly portable. There are client libraries for Cassandra, for almost any language platform in existence. In our case, we chose Pycassa Python module to interface Cassandra.
Data Design and migration scheme
Design of PanDA data for storage in the Cassandra system underwent a few iterations which included storage of data as csv (comma-separated values) data chunks, “bucketing” of data in groups with sequential identifiers (to facilitate queries of serially submitted jobs) etc. After performance evaluation, a simple scheme was adopted where a single job entry maps onto a Cassandra row. The row key then is the same as used as primary index in the Oracle database, which is the unique ID automatically assigned to each job by the PanDA system.
At the time of this writing, parts of the archived data in PanDA are mirrored on Amazon S3 system in csv format for backup and R\&D purposes, grouped in segments according to the date of the last update. This body of data was used as input for populating a Cassandra instance as it effectively allows us to avoid placing an extra load on the production RDBMS during testing, is globally and transparently available and does not require DB-specific tools to access data (all data can be downloaded by the client application from a specific URL using tools like curl, wget etc).
In this arrangement, a multithreaded client application (written in Python) pulls the data from Amazon, optionally caches it on the local file system, parses the csv format, and feeds it to the Cassandra cluster.
Characteristics of the cluster used in the project
In order to achieve optimal performance, Cassandra makes full use of the resources of the machine on which it is deployed. It is therefore not optimal to place a Cassandra system on a cluster of virtual machines, since there is nothing to be gained from sharing redundant resources. For that reason, after initial testing on a 4-VM cluster at CERN, a dedicated cluster was built at Brookhaven National Laboratory, with the following characteristics: 3 nodes with 24 cores, 48GB RAM and 1TB of attached SSD storage each. Its first version relied on rotational media (6 spindles per node in RAID0 configuration), but testing showed that it didn’t scale up to performance levels of the PanDA Oracle instance, therefore storage was upgraded to SSDs. Main results of testing will be given below. The Oracle cluster used in PanDA had a total spindle count of 110 (Raptor-class rotational media). Its performance would doubtless benefit from using SSDs as well, but clearly at this scale this wouldn’t be economical.
It’s worth noting that Cassandra use of resources is radically different between read and write operations – the former is I/O bound, while the latter is CPU intensive. Depending on application, it’s important to provide sufficient resources in each domain.
Monitoring load on the nodes as well as client machines that load and index data on the cluster is an important part of regular operations, troubleshooting and maintenance. We used the Ganglia system to accomplish that.
Indexing
Efficient use of data makes it necessary to create an effective indexing scheme in the data. Analysis of actual queries done against PanDA database shows that in addition to a simple row key query they often contain selection based in values in a few columns. This knowledge can be utilized in building a Cassandra-based application, by creating composite indexes mapped to such queries. This can be done in one of two ways:
1. Relying on “native” indexing in Cassandra
2. Creating additional lookup tables by a separate client application
If one goes with the former, it effectively necessitates creation of composite columns in the column families. This is akin to duplication of parts of data and leads to inflation of disk space associated with the data. On the other hand, there is no extra code to be written or run against the database, as the cluster will handle indexing transparently and asynchronously.
Producing indexes “by hand” (the second option) gives the operator complete control over when and how the data is indexed, and saves disk space since no extra columns are attached to column families. In the end, driven by consideration of referential integrity when retiring parts of data, it was decided to use the former option in creating indexes, trading disk space for ease of operation and maintenance.
Indexes were designed as follows: we identified queries most frequently used in the PanDA monitor. As an example, there are queries that select data based on the following job attributes: computingsite, prodsourcelabel, date. A new column is created in each row, with a name that indicates its composite nature: “computingsite+prodsourcelabel+date”. The value of such column may then look like ``ANALY\_CERN+user+20110601``. This augmentation of data can be done either at loading stage, or by running a separate process at a later time, at discretion of the operator. When and if this column is declared as index for Cassandra (which is typically done via its CLI), the cluster will start building corresponding data structures, automatically and asynchronously. To utilize the index, client application must determine that the user’s request contains requisite query elements, and then a query is run on the cluster by selecting values of the composite column according to the user’s request. Seven indexes constructed in such a way cover a vast majority of all queries done in the Monitor. The remaining small fraction can be handled by using individual “single” column indexes in combination, since this is also allowed in Cassandra. It won’t be as fast as a composite index since it will involve iterations over a collection of objects instead of a seek and fetch, however preliminary testing shows that the performance will still be acceptable, given relative scarcity of such queries.
Results of testing
A few client applications, written in Python, were created for purposes of data loading and indexing, as well as performance and scalability tests. The Cassandra cluster was populated with real PanDA data, and the data load was equal to one year worth of job records, covering the period of June 2010 to May 2011, in order to be approximately the same as in testing done with previous configurations. Based on observed query patterns, two main categories of performance and scalability tests were performed, along with comparisons with analogous queries against the original Oracle database:
Random seek test, whereby a long series of queries were extracting individual data for randomly selected jobs from the past year data sample
Indexed queries, when a number of representative complex queries were performed using composite indexes as described in the section above

In order to judge the scalability of the system, queries of both classes were run concurrently in multiple client threads, simulating real operational conditions of the database. Our capability to fully stress the cluster was limited to approximately 10 threads, by the available CPU resources on the client machine, however the results extracted provide useful guidance nevertheless.
In the random seek test, Cassandra’s time per extracted entry was 10 ms, with 10 concurrent clients. This translates into 1000 queries per second, which is about twice the rate currently experienced by our Oracle database. An analogous test done against Oracle (from a Python client) yielded roughly 100ms per query.
In the case of indexed queries, Cassandra’s time per entry was 4ms. This result is counterintuitive at first, when compared to the random query metric, because indexed queries result in additional disk operations (the index must be read before the actual data extraction). However this result makes sense  if one considers that multiple rows extracted in the query were packaged more efficiently in the network layer during data transmission from the cluster to the client, i.e. effective overhead per entry was lower. A comparable Oracle test returned a range of values from 0.5ms to 15 ms.
The queries that were run for comparison purposes against Oracle and Cassandra were effectively identical in scope and logic, i.e. full content of the row was pulled from the database in each case.
When increasing the load further, we found that the Cassandra cluster will adequately handle loads of at least 1500 queries per second, which is approximately 3 times more than current query load in the PanDA Monitor. Actual timing results were less relevant because of the client machine limitations mentioned above, i.e. we were not able to reach stress limits of the Cassandra cluster at this time.
Conclusions
We identified a “noSQL” database system Cassandra as a promising technology platform to ensure scalability and performance of the PanDA monitor database, operating under conditions of consistently high data and query load. A few versions of data design and indexing were evaluated. Hardware requirements were confirmed via testing of the Cassandra cluster under realistic conditions and benchmarking it against Oracle RDBMS. A simple but effective technique of composite index creation was employed to guarantee high performance in most popular queries generated in the PanDA monitor. Quantitative scalability and performance test results are favorable and pave the way for integration of this new noSQL data store with PanDA monitor.








%%%%%%%%%%%%%%%%%%%%


\section{Measuring radiation levels in Seattle}



Researchers at the University of Washington utilized the Cloudant NoSQL database as part of an experiment that determined radiation levels in Seattle as a result of the recent Fukushima nuclear disaster are “well below alarming limits.” The research team, which includes Cloudant Founder and Chief Scientist Mike Miller (his other title is research associate professor of physics), studied particles captured from the five air filters at the university’s Physics and Astronomy building and used Cloudant’s CouchDB-based BigCouch database to store and process the data from its experiments.

The research team's shielded germanium detector and data acquisition hardware.

The team’s results show radioactive particles first reached Seattle on March 17, but they were less abundant in volume and type than was expected based on research carried out after the Chernobyl disaster in 1986. In fact, the project’s official website states, “We stress that the overall amount of the radioactivity is extremely low, at least thousands of times below EPA limits.” Miller attributes the difference in radiation levels resulting from Chernobyl to the fact that the reactor at Chernobyl “was operating at full steam when it exploded,” whereas his team’s research suggests the Fukushima plant’s automatic safety system must have kicked in when the earthquake hit, turning off the reactor in the process.

Miller’s association with Cloudant certainly influenced his team’s decision to use the product, but BigCouch — which is more big-data-focused than its web-application-focused NoSQL counterparts — is particularly well-suited for this type of job. According to Miller, the team created “a MacGuyver-like setup” in order to start monitoring radiation levels in a hurry. After installing the special air filters and sampling “gigantic quantities of air,” the trapped particles were run through a germanium detector to determine the unique fingerprints of isotopes created by nuclear fission. Miller says the general belief is that radioactive isotopes travel through the air attached to dust particles or concrete particles from the explosion.

Cloudant’s BigCouch database let the team keep up with a steady flow of data so it could process  and analyze it, then share it with the various stakeholders in near-real-time. The team was changing the data about 20 times per day and writing complex workflows to process it, two tasks that fall into BigCouch’s wheelhouse. The database has a built-in MapReduce engine to enable writing and processing the workflows, and it allows for secondary indices, which users can populate with new data from their MapReduce jobs and query very quickly. Miller actually helped create Cloudant in 2008 while doing post-doctoral research at MIT that involved analyzing huge data sets from CERN’s Large Hadron Collider.

Although humans might not be at risk from the Fukushima-based radiation, their scientific experiments might not be so lucky. In an article by the University of Washington’s Office of News and Information, Miller’s colleague R.G. Hamish Robertson explained that the radiation levels “can raise havoc with sensitive physics experiments. That includes one called Majorana, in which the UW physicists are deeply involved, that is being planned for a lab nearly [one] mile down in the proposed Deep Underground Science and Engineering Laboratory in the old Homestake Mine in Lead, S.D.” Miller told me that project, which involves trying to determine what the universe is made of, has been suspended because the next step is to bring highly sensitive homegrown copper up from underground for machining, but that even the trace levels of radiation present in South Dakota could taint it and ruin the experiment.

The currently-available paper sharing the team’s results and insights mentions some uncertainty based on questions about the particle sizes they were trying to capture, but Miller said his team has made progress on this front and that they’re “very confident” in the results. The team will release a final paper soon with more detail and a larger time frame of data points.

\chapter{Conclusions and future work}


\section{Conclusions}

\begin{itemize}

\item Data generated by huge scientific projects are becoming a problem.

\item Relational approach are not always suitable for any problem. Non-relational systems are not cure-all, but it has been shown they can face some problems in a more efficient way (\textit{e.g.} MapReduce).

\item Any new proposal should be inside Virtual Observatory frame.

\item NoSQL database systems, specially -not exclusively- those document-oriented can reduce system analysis and design and can also succeed in boosting the performance of data management.

\end{itemize}

\section{Future work}

\begin{itemize}

\item It would be desirable to focus in a work group inside Virtual Observatory instead of treating several aspects.

\item It is highly recommended to use formal metrics to measure the effectiveness of systems rewriting/redesigning.

\item In order to obtain accurate data in performance improvements benchmarks must be used.

\item It would be of interest to decide which language to use to connect the selected DBMS.

\end{itemize}




\appendix
%% Cap'itulos incluidos despues del comando \appendix aparecen como ap'endices
\include{apendiceBC}
\include{apendiceC}

\nocite{*}
\bibliography{biblio}

\end{document}
