
\chapter{The Virtual Observatory}

The International Virtual Observatory Alliance (IVOA) was formed in June 2002 with a mission to "facilitate the international coordination and collaboration necessary for the development and deployment of the tools, systems and organizational structures necessary to enable the international utilization of astronomical archives as an integrated and interoperating virtual observatory." The IVOA now comprises 20 VO programs from Argentina, Armenia, Australia, Brazil, Canada, China, Europe, France, Germany, Hungary, India, Italy, Japan, Russia, Spain, the United Kingdom, Ukraine, and the United States and inter-governmental organizations (ESA and ESO). Membership is open to other national and international programs according to the IVOA Guidelines for Participation.\\

The IVOA focuses on the development of standards and encourages their implementation for the benefit of the worldwide astronomical community. Working Groups are constituted with cross-program membership in those areas where key interoperability standards and technologies have to be defined and agreed upon. The Working Groups develop standards using a process modeled after the World Wide Web Consortium, in which Working Drafts progress to Proposed Recommendations and finally to Recommendations. Recommendations may ultimately be endorsed by the Virtual Observatory Working Group of Commission 5 (Astronomical Data) of the International Astronomical Union. The IVOA also has Interest Groups that discuss experiences using VO technologies and provide feedback to the Working Groups. Ad-hoc and permanent committees deal with specific scientific and procedural topics. Interaction with other scientific disciplines interested in data inter-operability is also pursued through dedicated Liaison Groups.\\

Senior representatives from each national and international member VO program form the IVOA Executive Committee. A chair is chosen from among the representatives and serves an eighteen-month term, normally preceded by an eighteen-month term as deputy chair. The Executive Committee meets 3-4 times a year (also by teleconference) to discuss goals, priorities, and strategies. Executive Committee members represent their respective programs and are expected to be in a position to commit resources targeted at the achievement of common goals. Decisions by the IVOA Executive Committee are reached by consensus.\\

The IVOA holds two Interoperability Workshops each year typically in May and October. These meetings are opportunities for the IVOA Groups and Committees to have face-to-face discussions.

\section{ObsTAP}

In 2011 IVOA proposed a new recommendation: Observation Data Model Core Components and its Implementation in the Table Access Protocol. That document was intended to be a description of the interface which integrated the data modeling and data access aspects in a single service:\\ 

\framebox{ObsCore data model + Table Access Protocol = ObsTAP}


\section{FITS format}
 
Flexible Image Transport System (FITS) is an open standard defining a digital file format useful for storage, transmission and processing of scientific and other images. FITS is the most commonly used digital file format in astronomy. Unlike many image formats, FITS is designed specifically for scientific data and hence includes many provisions for describing photometric and spatial calibration information, together with image origin metadata.\\

The FITS format was first standardized in 1981; it has evolved gradually since then, and the most recent version (3.0) was standardized in 2008. FITS was designed with an eye towards long-term archival storage, and the maxim once FITS, always FITS represents the requirement that developments to the format must be backwards compatible.\\
 
A major feature of the FITS format is that image metadata is stored in a human-readable ASCII header, so that an interested user can examine the headers to investigate a file of unknown provenance. The information in the header is designed to calculate the byte offset of some information in the subsequent data unit to support direct access to the data cells. Each FITS file consists of one or more headers containing ASCII card images (80 character fixed-length strings) that carry keyword/value pairs, interleaved between data blocks. The keyword/value pairs provide information such as size, origin, coordinates, binary data format, free-form comments, history of the data, and anything else the creator desires: while many keywords are reserved for FITS use, the standard allows arbitrary use of the rest of the name-space.\\
 
FITS is also often used to store non-image data, such as spectra, photon lists, data cubes, or even structured data such as multi-table databases. A FITS file may contain several extensions, and each of these may contain a data object. For example, it is possible to store x-ray and infrared exposures in the same file.\\
 
FITS support is available in a variety of programming languages that are used for scientific work, including C, C++, C\#, Fortran, IGOR Pro, IDL, Java, LabVIEW, Mathematica, MatLab, Perl, PDL, Python, R, and Tcl. The FITS Support Office at NASA/GSFC maintains a list of libraries and platforms that currently support FITS.\\
 
Image processing programs such as ImageJ, GIMP, Photoshop, XnView and IrfanView can generally read simple FITS images, but frequently cannot interpret more complex tables and databases. Scientific teams frequently write their own code to interact with their FITS data, using the tools available in their language of choice. The FITS Liberator software is used by imaging scientists at the European Space Agency, the European Southern Observatory and NASA. The SAOImage DS9 Astronomical Data Visualization Application is available for many OSs, and handles images and headers.\\

%% Los cap'itulos inician con \chapter{T'itulo}, estos aparecen numerados y
%% se incluyen en el 'indice general.
%%
%% Recuerda que aqu'i ya puedes escribir acentos como: 'a, 'e, 'i, etc.
%% La letra n con tilde es: 'n.

%\chapter{OpenCADC}
\section{OpenCADC}
Esto es una prueba